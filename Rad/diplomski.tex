\documentclass[times, utf8, diplomski]{fer}
\usepackage{booktabs}

\begin{document}

% TODO: Navedite broj rada.
\thesisnumber{2451}

% TODO: Navedite naslov rada.
\title{Klasifikacija pokreta ljudskog tijela temeljena na podacima s inercijskih senzora}

% TODO: Navedite vaše ime i prezime.
\author{Ivan Trubić}

\maketitle

% Ispis stranice s napomenom o umetanju izvornika rada. Uklonite naredbu \izvornik ako želite izbaciti tu stranicu.
\izvornik

% Dodavanje zahvale ili prazne stranice. Ako ne želite dodati zahvalu, naredbu ostavite radi prazne stranice.
\zahvala{}

\tableofcontents

\chapter{Uvod}
Karakteristike pokreta ljudskoga tijela vrlo su individualne te ovise o mnogo čimbenika kao što su genetika, odgoj te
fizička sprema. Ti pokreti su toliko jedinstveni da se mogu koristiti za identifikaciju osoba dok su s druge strane
toliko slični da čak i manje devijacije u tim pokretima također mogu ukazivati na neke zdravstvene probleme.
Ljudski pokreti mogu se klasificirati na razne načine koristeći računalni vid ili razne senzore postavljene na
ljudskome tijelu. Svaka od metoda ima svoju granu primjene kao: identifikacija osoba temeljenog na hodu koristeći
računalni vid na nadzornim kamerama \citep{surveillance}, pračenje pokreta igrača u interakciji sa igrama u virtualnoj stvarnosti \citep{VR},
korištenje inercijskih (IMU) senzora za precizno snimanje hoda u svhu otkrivanja bolesti i rehabilitacije te mnoge druge.

Klasifikacija pokreta vrlo je složen problem te kao takav nema dobro rješenje koristeći klasične algoritme. Razvojem moči računala
te metoda strojnoga učenja ovaj problem postaje rješiv. Za snimanje pokreta može se koristiti kamera ili senzori.
Koristeći kameru, na snimci se koriste metode računalnog vida te se traže karakteristike ljudskog tijela kako bi se na snimci
prepoznala osoba te koristeći te karakteristične točke analizira se hod. Također, kamera može snimati osobu sa posebno postavljenim
vizualnim oznakama po djelovima tijela te koristeći te vizualne oznake analizirati pokrete. Nedostatak kamera je taj što snimaju iz
jedne perspektive te zbog toga može doći do okluzije oznaka. Inercijski (IMU) senzori eliminiraju kamere te ne pate od problema okluzije.
Inercijski senzori su relativno jeftini i mali uređaji koji se postave na ključne djelove ljudskoga tijela te pružaju vrlo dobar uvid
u ljudske pokrete. Primjerice mogu se staviti na ruke te upravljati igrama i uređajima ali se mogu koristiti i u medicinske svrhe za
analizu hoda i diagnosticiranje zdravstvenih problema kao i za provođenje terapijskih vježbi bez nadzora stručnjaka.
Ovaj rad će se više fokusirati na medicinski aspekt klasifikacije pokreta, preciznije analizu hoda (\textit{eng.} gait) i terapiju koljena.



\chapter{Razrada}

\section{Bolesti koljena}

\section{Dosadašnja rješenja}

\section{Rješenje koristeći IMU senzore}

\section{Analiza IMU senzora}

\section{Analiza dostupnih baza podataka}

\section{Stvaranje vlastite baze podataka}

\section{Implementacija metode strojnoga učenja}

\section{Rezultati}

\chapter{Zaključak}
Zaključak.

\bibliography{literatura}
\bibliographystyle{fer}

\begin{sazetak}
Sažetak na hrvatskom jeziku.

\kljucnerijeci{Ključne riječi, odvojene zarezima.}
\end{sazetak}

% TODO: Navedite naslov na engleskom jeziku.
\engtitle{Title}
\begin{abstract}
Abstract.

\keywords{Keywords.}
\end{abstract}

\end{document}
